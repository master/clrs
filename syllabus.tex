\documentclass[a4paper,10pt]{article}
\usepackage[utf8]{inputenc}
\usepackage[russian]{babel}
\begin{document}
\section*{Алгоритмы: построение и анализ}
\begin{enumerate}
\item Анализ алгоритмов. Сортировка вставкой и сортировка слиянием
\item Корректность алгоритма. Схема Горнера
\item Асимптотическая нотация. Рекуррентные соотношения
\item Парадигма ``Разделяй и властвуй'': алгоритм Штрассена, алгоритм Фибоначчи, умножение матриц
\item Сортировка Quicksort. Рандомизированные алгоритмы
\item Пирамидальная сортировка. Очереди с приоритетами
\item Сортировка за линейное время: нижние оценки, сортировка подсчётом, поразрядная сортировка
\item Медианы и порядковые статистики
\item Применение медиан. Блочная сортировка
\item Хэширование. Хэш-функции
\item Универсальное хэширование. Идеальное хэширование
\item Двоичные поисковые деревья. Обход дерева
\item Связь поискового дерева и Quicksort. Анализ рандомизированного алгоритма.
\item Красно-черные деревья. Алгоритмы вращения, вставки, удаления
\item 2-3 деревья. B-деревья
\item Расширение структур данных. Динамические порядковые статистики. Деревья отрезков
\item Деревья расстояний
\item Списки с пропусками (слоёные списки)
\item Амортизационный анализ. Динамические таблицы. Метод потенциалов
\item Конкурентный анализ: самоупорядочивающиеся списки
\item Конкурентный анализ: задача аренды лыж
\item Динамическое программирование. Задача о наибольшей общей подпоследовательности
\item Жадные алгоритмы. Минимальные остовные деревья
\item Кратчайшие пути: свойства, алгоритм Дейкстры, поиск в ширину
\item Кратчайшие пути: алгоритм Беллмана-Форда, линейное программирование, ограничения на разности
\item Поиск в глубину в графе. Топологическая сортировка
\item Кратчайшие пути: все пары вершин, умножение матриц, алгоритм Флойда-Уоршолла, алгоритм Джонсона
\end{enumerate}
\end{document}
