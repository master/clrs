\documentclass[a4paper,11pt]{article}
\usepackage[utf8]{inputenc}
\usepackage[russian]{babel}
\usepackage{hyperref}

\author{Олег Смирнов, Иван Веселов \\
\texttt{oleg.smirnov@gmail.com, veselov@gmail.com}}
\date{2009/06/04}
\title{Алгоритмы: построение и анализ -- программа курса}

\begin{document}
\maketitle
\begin{abstract}
Программа базируется на курсе ``6.046: Introduction to Algorithms'' Массачусетского
технологического института\footnote{\href{http://goo.gl/jIOiq}
{http://ocw.mit.edu/ ... /6-046j-introduction-to-algorithms-sma-5503-fall-2005}} и
на одноимённом учебнике Т. Кормена, Ч Лейзерсона, Р. Ривеста и К. Штайна.
\end{abstract}
\section*{Раздел 1. Анализ алгоритмов}
\begin{itemize}
\item Тема 1.1. Анализ алгоритмов. Сортировка вставкой и сортировка слиянием
\item Тема 1.2. Корректность алгоритма. Схема Горнера
\item Тема 1.3. Асимптотическая нотация. Рекуррентные соотношения
\item Тема 1.4. Парадигма ``Разделяй и властвуй'': алгоритм Штрассена, алгоритм Фибоначчи, умножение матриц
\item Тема 1.5. Сортировка Quicksort. Рандомизированные алгоритмы
\item Тема 1.6. Пирамидальная сортировка. Очереди с приоритетами
\item Тема 1.7. Сортировка за линейное время: нижние оценки, сортировка подсчётом, поразрядная сортировка
\item Тема 1.8. Медианы и порядковые статистики
\item Тема 1.9. Применение медиан. Блочная сортировка
\end{itemize}
\section*{Раздел 2. Разработка алгоритмов}
\begin{itemize}
\item Тема 2.1. Хэширование. Хэш-функции
\item Тема 2.2. Универсальное хэширование. Идеальное хэширование
\item Тема 2.3. Двоичные поисковые деревья. Обход дерева
\item Тема 2.4. Связь поискового дерева и Quicksort. Анализ рандомизированного алгоритма.
\item Тема 2.5. Красно-черные деревья. Алгоритмы вращения, вставки, удаления
\item Тема 2.6. 2-3 деревья. B-деревья
\item Тема 2.7. Расширение структур данных. Динамические порядковые статистики. Деревья отрезков
\item Тема 2.8. Деревья расстояний
\item Тема 2.9. Списки с пропусками (слоёные списки)
\item Тема 2.10. Амортизационный анализ. Динамические таблицы. Метод потенциалов
\item Тема 2.11. Конкурентный анализ: самоупорядочивающиеся списки
\item Тема 2.12. Конкурентный анализ: задача аренды лыж
\item Тема 2.13. Динамическое программирование. Задача о наибольшей общей подпоследовательности
\end{itemize}
\section*{Раздел 3. Алгоритмы на графах}
\begin{itemize}
\item Тема 3.1. Жадные алгоритмы. Минимальные остовные деревья
\item Тема 3.2. Кратчайшие пути: свойства, алгоритм Дейкстры, поиск в ширину
\item Тема 3.3. Кратчайшие пути: алгоритм Беллмана-Форда, линейное программирование, ограничения на разности
\item Тема 3.4. Поиск в глубину в графе. Топологическая сортировка
\item Тема 3.5. Кратчайшие пути: все пары вершин, умножение матриц, алгоритм Флойда-Уоршолла, алгоритм Джонсона
\end{itemize}
\end{document}
