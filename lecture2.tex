\documentclass[11pt]{article}
\usepackage[utf8]{inputenc}
\usepackage[russian]{babel}
\usepackage[T1]{fontenc}
\usepackage{amssymb,amsmath,clrscode,graphicx,indentfirst}

\author{Олег Смирнов}
\title{Курс kiev-clrs -- Лекция 2. Асимптотическая нотация}
\date{17 января 2009 г.}

\begin{document}
\maketitle
\tableofcontents

\newpage
\setlength{\parskip}{1ex plus 0.5ex minus 0.2ex}
\section{Цель лекции}
\begin{itemize}
\item Ввести строгие математические определения для $O$-, $\Omega$- и $\Theta$-нотации 
\item Рассмотреть методы решения рекурентностей: подстановка, дерево рекурсии и основной метод
\end{itemize}

\section{Асимптотическая нотация}

\subsection{$O$-обозначения}
В случае, когда необходимо определить только \emph{асимптотическую верхнюю границу}, используют $O$-обозначения:
\begin{equation*}
  f(n) = O(g(n)) \Rightarrow \exists c, n_0 > 0 \text{ такие, что }
  0 \leqslant f(n) \leqslant c g(n), \forall n > n_0
\end{equation*}

$O(g(n))$ можно рассматривать как множество функций:
\begin{equation*}
  O(g(n)) = \{f(n): \exists c, n_0 > 0 \text{ такие, что }
  0 \leqslant f(n) \leqslant c g(n), \forall n > n_0
  \}
\end{equation*}

Пример:
\begin{equation*}
  2n^3 = O(n^3) \text{ для } c = 1, n_0 = 2 \text{ или } 2n^3 \in O(n^3)
\end{equation*}

Определение через множество можно использовать в качестве ``макроса''
-- $O$-нотация в формуле обозначает некоторую функцию из соответствующего семейства.

Пример:
\begin{equation*}
  n^2 + O(n) = O(n^2) \text{ означает } \forall f(n) \in O(n): \exists h(n) \in O(n^2): n^2 + f(n) = h(n)
\end{equation*}

\subsection{$\Omega$-обозначения}

Для определения \emph{асимптотическую нижней границы} есть $\Omega$-обозначение:
\begin{equation*}
  \Omega(g(n)) = \{f(n): \exists c, n_0 > 0 \text{ такие, что }
  0 \leqslant c g(n) \leqslant f(n), \forall n > n_0
  \}
\end{equation*}

Пример:
\begin{equation*}
  \sqrt(n) = \Omega(lg n) \text{ для } c = 1, n_0 = 16
\end{equation*}

\subsection{$\Theta$-обозначения}

Для точной оценки используется $\Theta$-обозначение. Его можно ввести несколькими способами:
\begin{itemize}
\item Теорема: для любых двух функций $f(n)$ и $g(n)$ соотношение $f(n)~=~\Theta(g(n))$ верно тогда и только тогда, когда $f(n)~=~O(g(n))$ и $f(n)~=~\Omega(g(n))$
\item Пересечение множеств: $f(n)~=~\Theta(g(n))$ $\iff$ $f(n)~=~O(g(n))~\cap~\Omega(g(n))$
\end{itemize}

Пример:
\begin{equation*}
  {1 \over 2}n^2 - 2n = \Theta(n^2)
\end{equation*}

\subsection{$o$- и $\omega$-обозначения}

$o$- и $\omega$-обозначения являются версиями определений $O$ и $\Omega$, которые выполняются 
\emph{для любой константы $c$}, т.е. не являются асимптотически строгими. Формально:
\begin{equation*}
  o(g(n)) = \{f(n): \forall c > 0: \exists n_0 > 0 \text{ такое, что }
  0 \leqslant f(n) \leqslant c g(n), \forall n > n_0
  \}
\end{equation*}
Функция $f(n)$ пренебрежима мала по сравнению с функцией $g(n)$ при $n$ стремящемся к бесконечности, т.е.: 
\begin{equation*}
  \lim_{x\to\infty} { f(x) \over g(n) } = 0
\end{equation*}
Пример:
\begin{equation*}
  2n^2 = o(n^3) \text{ для } n_0 = {2 \over c}
\end{equation*}
Аналогично:
\begin{equation*}
  \omega(g(n)) = \{f(n): \forall c > 0: \exists n_0 > 0 \text{ такое, что }
  0 \leqslant c g(n) \leqslant f(n), \forall n > n_0
  \}
\end{equation*}
Пример:
\begin{equation*}
  \sqrt(n) = \omega(lg n) \text{ для } n_0 = 1 + {1 \over c}
\end{equation*}

\subsection{Некоторые свойства}

Из определений вытекают некоторые свойства асимтотических сравнений:

\begin{itemize}
\item Транзитивность
\begin{equation*}
  f(n) = \Theta(g(n)) \land g(n) = \Theta(h(n)) \Rightarrow f(n) = \Theta(h(n))
\end{equation*}
\begin{equation*}
  f(n) = O(g(n)) \land g(n) = O(h(n)) \Rightarrow f(n) = O(h(n))
\end{equation*}
\begin{equation*}
  f(n) = \Omega(g(n)) \land g(n) = \Omega(h(n)) \Rightarrow f(n) = \Omega(h(n))
\end{equation*}
\begin{equation*}
  f(n) = o(g(n)) \land g(n) = o(h(n)) \Rightarrow f(n) = o(h(n))
\end{equation*}
\begin{equation*}
  f(n) = \omega(g(n)) \land g(n) = \omega(h(n)) \Rightarrow f(n) = \omega(h(n))
\end{equation*}

\item Рефлексивность
\begin{equation*}
  f(n) = \Theta(f(n))
\end{equation*}
\begin{equation*}
  f(n) = O(f(n))
\end{equation*}
\begin{equation*}
  f(n) = \Omega(f(n))
\end{equation*}

\item Симметричность
\begin{equation*}
  f(n) = \Theta(g(n)) \iff g(n) = \Theta(f(n))
\end{equation*}

\item Перестановочная симметрия
\begin{equation*}
  f(n) = O(g(n)) \iff g(n) = \Omega(f(n))
\end{equation*}
\begin{equation*}
  f(n) = o(g(n)) \iff g(n) = \omega(f(n))
\end{equation*}
\end{itemize}

Таким образом можно провести аналогию между символами нотации и операциями
сравнения рациональных чисел:
\begin{equation*}
  f(n) = \Theta(g(n)) \approx f(n) = g(n)
\end{equation*}
\begin{equation*}
  f(n) = O(g(n)) \approx f(n) \leqslant  g(n)
\end{equation*}
\begin{equation*}
  f(n) = \Omega(g(n)) \approx f(n) \geqslant g(n)
\end{equation*}
\begin{equation*}
  f(n) = o(g(n)) \approx f(n) < g(n)
\end{equation*}
\begin{equation*}
  f(n) = \omega(g(n)) \approx f(n) > g(n)
\end{equation*}

\section{Рекуррентные соотношения}

По определению, рекуррентное соотношение -- это уравнение или неравенство,
описывающее функцию с использованием её самой, но только с меньшими аргументами.
Обычно рекуррентное соотношение описывается в виде системы граничных условий
и формулы для общего случая, например:

\begin{equation*}
  T(n) = \begin{cases}
    \Theta(1), \text{ если } n = 1 \\
    2\Theta(n/2) + \Theta(n), \text{ если } n > 1
    \end{cases}
\end{equation*}

Для решения таких соотношений использутся несколько методов: подстановки, деревьев
рекурсии и основной метод.

\section{Метод подстановки}

Метод состоит из трех шагов:
\begin{itemize}
\item делается догадка о виде решения
\item с помощью метода математической индукции доказывается, что решение правильное
\item вычисляются константы 
\end{itemize}
Пример:

\section{Метод деревьев рекурсии}

\section{Основной метод}


\end{document}
