\documentclass[11pt]{article}
\usepackage[utf8]{inputenc}
\usepackage[russian]{babel}
\usepackage[T1]{fontenc}
\usepackage{amssymb,amsmath,clrscode,graphicx,indentfirst}

\author{Олег Смирнов}
\title{Курс kiev-clrs -- Лекция 6. Медианы и порядковые статистики}
\date{16 мая 2009 г.}

\begin{document}
\maketitle
\tableofcontents

\newpage
\setlength{\parskip}{1ex plus 0.5ex minus 0.2ex}
\section{Цель лекции}
\begin{itemize}
\item Дать понятие $i$-й порядковой статистики и медианы массива
\item Алгоритм выбора порядковой статистики за линейное ожидаемое время
\item Алгоритм выбора за линейное время в наихудшем случае
\end{itemize}

\section{Введение}

Профессор работает консультантом в нефтяной компании, которая запланирует провести магистральный трубопровод от восточного до западного края нефтяного месторождения с $n$ скважинами. От каждой скважины к магистральному трубопроводу кратчайшим путем проведены рукава. Каким образом профессор может выбрать оптимальное расположение трубопровода (т.е. такое, при котором общая длина всех рукавов была бы минимальной) по заданым координатам скважин ($x$, $y$)?

Легко видеть, что в случае чётного количества скважен $n$, трубопровод можно провести в любом месте при условии, что по обе его стороны (с севера и с юга) будет равное количество скважин. Множества скважин можно представить в виде массива их $y$ координат, отсортированного по возрастанию элементов. Условие будет достигнуто, если трубопровод проходит между скважинами с номерами $n/2$ и $n/2+1$ в указанном массиве.

В случае нечётного количества, можно найти скважину, координата $y$ которой лежит ``в центре'' массива, т.е. с порядковым номером $\frac{n+1}{2}$. Если провести трубопровод через указанную скважину, то растояние до неё будет равно нулю и задача сведется к предыдущему случаю.

\section{Определение}

Будем называть $i$-й порядковой статистикой множества, состоящего из $n$ элементов, $i$-й элемент в порядке возратания. Например, минимум такого множества -- это первая порядковая статистика ($i = 1$), а его максимум -- это $n$-я порядковая статистика. Медиана неформально обозначает середину множества. Если $n$ нечётное, то медиана единственная, и её индекс равен $i = \frac{n+1}{2}$; если же $n$ чётное, то медианы две, и их индексы равны $i=n/2$ и $i=n/2 +1$. Таким образом, независимо от чётности $n$, медианы располагаются при $i = \lfloor \frac{n+1}{2} \rfloor$ (нижняя медиана) и $i = \lceil \frac{n+1}{2} \rceil$ (верхняя медиана).

\end{document}
